\documentclass[a5paper,twoside]{article}
\usepackage[utf8]{inputenc}
\usepackage{lmodern}
\usepackage[T1]{fontenc}
\usepackage[hungarian]{babel}

\usepackage{graphicx}
\graphicspath{ {images/} }

\usepackage[bookmarks]{hyperref}

% \usepackage[chorded]{songs}
\usepackage[lyric]{songs}
% \usepackage[slides]{songs}

\title{Gitáros mise\\\textit{(2021. január 02.)}}
% \author{Donkó István}
\date{}

\setlength{\topmargin}{-1.5cm}

\setlength{\headheight}{0cm}
\setlength{\headsep}{0cm}
\setlength{\footskip}{0.85cm}

\setlength{\oddsidemargin}{-1.3cm}
\setlength{\evensidemargin}{-1.8cm}

\setlength{\textwidth}{12.9cm}
\setlength{\textheight}{18.5cm}

\setlength{\songnumwidth}{1.02cm}
\setlength{\versenumwidth}{0.5cm}
% \setlength{\cbarwidth}{0.1cm}

\catcode`_=12 % use underscore as regular character
\renewcommand{\_}[1]{\underline{#1}} % shorter command for adding melismas
\songcolumns{1}
\notenames{A}{H}{C}{D}{E}{F}{G}

\begin{document}
  \pagenumbering{arabic}

  % \begin{titlepage}
  %   \pagenumbering{gobble}
    \setlength{\oddsidemargin}{-1.625cm}

  %   \vspace*{4cm}
    {\let\newpage\relax\maketitle}
  % \end{titlepage}

  \versesep=12pt plus 3pt minus 5pt

  \iflyric
    \baselineadj=2pt plus 1pt minus 1pt
  \fi

  % \includeonlysongs{M11,101,56,M11,101,56}
  % \includeonlysongs{80,M10,55,26,M30,M40,85,40,15}
  % \includeonlysongs{46,M10,M20,84,20,56,M30,M40,87,48,97,73,40,92}
  \includeonlysongs{46,M10,M20,T1,T101,24,M30,M41,48,50,73,92}
  % \includeonlysongs{T101}
  % \includeonlysongs{999}
  % \includeonlysongs{999,M41}
  % \includeonlysongs{54,90,36,80,91,77,97,40,84,11,6}

  \begin{songs}{}
    \input{songs.tex}
    \setcounter{songnum}{101}

\beginsong{Áldom szent neved}
[
  by={Matt Redman, Beth Redman, Vass Tímea}
]

  \beginverse
    \[A] Áldom \[E]szent neved a \[F#m]mézzel folyó \[D]földeken,
    Hol \[A]nincs hiányom \[E]semmiben, \[D]áldom nevedet!
    \[A] Áldom \[E]szent neved, ha az \[F#m]út a pusztán \[D]át vezet,
    A \[A]száraz, kiet\[E]len helyen \[D]áldom nevedet!
  \endverse

  \beginchorus
    \[A] Minden áldást \[E]dicséretté \[F#m] formál szí\[D]vem,
    \[A] Ha körülzár \[E]a sötétség \[F#m] azt mondom é\[D]n:

    Hogy áldott legye\[A]n az Úr neve, \[E]
    Áldom neve\[F#m]det, \[D]
    Áldott le\[A]gyen az Úr neve, \[E]
    Áldott legyen \[F#m]hatal\[E]mas neve\[D]d!
  \endchorus

  \beginverse
    ^ Áldom ^szent neved, ha ^éppen minden ^rendben megy,
    Ha ^napfényes az ^életem, ^áldom nevedet!
    ^ Áldom ^szent neved, a ^szenvedések ^útján is,
    Bár ^könnyek között ^áldozom, ^áldom nevedet!
  \endverse

  \beginchorus
    \[A] Minden áldást \[E]dicséretté \[F#m] formál szí\[D]vem,
    \[A] Ha körülzár \[E]a sötétség \[F#m] azt mondom é\[D]n:

    /: Áldott legye\[A]n az Úr neve, \[E]
    Áldom neve\[F#m]det, \[D]
    Áldott le\[A]gyen az Úr neve, \[E]
    Áldott legyen \[F#m]hatal\[E]mas neve\[D]d! :/ \rep{2}
  \endchorus

  \beginverse*
    /: Te \[A]adsz és elve\[E]szel, Te \[F#m]adsz és elve\[D]szel.
    A \[A]szívem így fe\[E]lel, hogy \[F#m]áldom \[E]szent ne\[D]ved! :/ \rep{2}
  \endverse

  \beginverse*
    Refrén
  \endverse

\endsong

\beginsong{Atya, áldott legyél}
[
  by={zene és szöveg: A. Fleury, adaptálta: Fábry K.}
]

  \beginchorus
    Atya, \[G]ál\[D]dott legy\[Em]él!
    Fiú, \[Am]téged i\[C]mádjon a \[Am]föld és az \[D]ég!
    Lélek \[G]á\[D]radj re\[Em]ánk!
    Örök \[Am]Iste\[D]nünk, Téged \[C]áld az i\[G]mánk.
  \endchorus

  \beginverse\memorize
    \[Em]Jó Aty\[C]ánk, örök \[D]Alkotónk, \[C]legyen meg az \[D]akara\[G]tod!
    Amint a \[Em]mennyben a te \[Am]szentje\[D]id, \[H7]dicsérünk \[Em]Téged itt a \[Am]földön \[D]is.
  \endverse

  \refrain

  \beginverse
    ^Jézu^sunk, a mi ^Pártfogónk, ^Te vagy a mi ^Szabadí^tónk,
    Egyszü^lött, kit nekünk ^ad az A^tya, ^jöjj, U^runk, ma hívunk: ^Marana ^tha!
  \endverse

  \refrain

  \beginverse
    ^Szállj le ^ránk, Lélek^istenünk, ^az utunkon ^légy ve^lünk!
    Jöjj, ve^zess, Te vagy az ^Élte^tőnk, ha ^fára^dunk, kérünk ^adj e^rőt!
  \endverse

  \refrain[2]

\endsong

\beginsong{Atyaisten, a néped}
[
  by={zene és szöveg: A. Broeders, adaptálta: Fábry K.}
]

  % https://youtu.be/TuE9nzUoTQA

  \beginverse
    Atya\[G]isten a \[Hm]néped jön \[C]Hozz\[Em\⁷⁄H]ád,
    És \[Am]el\[G⁄H]hozza \[C\²]ál\[C]doza\[D\⁴]t\[D]át,
    Mely a \[G]munkánk s a \[Hm]földünk gyü\[C]mölcs\[Em]e:
    \[Am]A szín\[G⁄H]bor \[C]és \[D\⁷]a ke\[G]nyér.
  \endverse

  \beginverse
    Az ^oltárra ^tesszük most ^vél^ük
    A ^di^cséret ^ál^doza^t^át,
    Vigye ^angyalod ^Hozzád az ^égb^e
    ^Fölsé^ged ^szí^ne e^lé.
  \endverse

  \beginverse
    A Te ^Szentlelked ^által lesz ^ért^ünk
    Ez a ^ke^nyér és ^bor ^éte^l^ünk.
    Az ^Oltári^szentségben ^Kriszt^us
    ^Eljön, ^és ^üd^vö^zít.
  \endverse

\endsong

\beginsong{Bárányom, bárányom}[]

  \ifchorded
    \beginverse*
      {\nolyrics
        Előjáték: \[G] \[Am] \[Em] \[C] \[G-Em] \[Am-D] \[G] \[D]
        \hspace{1.42cm} \[G] \[Am] \[Em] \[C] \[G-Em] \[Am-D] \[G]
      }
    \endverse
  \fi

  \beginverse\memorize
    \[G]Bárányom, \[Am]Bárányom, \[Em]Istennek B\_{á}\[C]ránya,
    \[G]Könyörülj \[Em]rajtam, \[Am]hadd jussak \[D]jobb sors\[G]ra! \[D]
    \[G] Adj nekem \[Am] nagy hitet, \[Em] amíg le\[C]{h\_{e}}t!
    \[G]Formáld \[Em]kedved sze\[Am]rint a \[D]szív\_{e}\[G]met!
  \endverse

  \beginchorus
    Én U\[Em]ram, add meg nekem,
    Jó U\[Am]ram, add meg nekem,
    \[D]Hadd legyen Mel\[D\⁷]letted majd a \[G]helyem!
    Lesem \[Em]minden vágyadat, tiszte\[Am]lem házadat.
    Én U\[F#]ram, jó Uram, add meg ne\[Am]k\[D]{\_{e}}m!
  \endchorus

  \beginverse
    ^ Én Uram, ^ adj egy kis ^ időt ne^kem,
    ^Vékony kis ^szalm\_{a}^szál ^az én hi^tem! ^
    ^ De ha Te ^ segítesz, ^ erős le^szek,
    ^Kertedként ^áp{\_{o}}^lom a ^lelke^met.
  \endverse

  \refrain

  \ifchorded
    \beginverse*
      {\nolyrics
        Utójáték: \[G] \[Am] \[Em] \[C] \[G-Em] \[Am-D] \[G] \[D]
        \hspace{1.49cm} \[G] \[Am] \[Em] \[C] \[G-Em] \[Am-D] \[G]
      }
    \endverse
  \fi

\endsong

\beginsong{Béke Istene}
[
  by={Pipó József}
]

  \beginverse
    \[Em]Béke \[C]Istene \[G]szállj \[D]ránk, \[Am]jöjj el, \[C]Mennyei \[H7]Bárány!
    \[Em]Szíved \[C]várja a \[G]szí\[D]vem, \[C]Jézus, \[H7]légy mene\[Em]dékem!
  \endverse

\endsong

\beginsong{Bemegyek a Szentek Szentjébe \\ For Your Name is Holy}
[
  by={Jim Cowan, fordította: Flach Ferenc}
]

  \beginverse
    \[Em]Bemegyek a Szentek Szent\[D]jébe,
    \[C]Bemegyek a \[D]Bárány vérén \[Em]át.
    \[Em]Bemegyek, csak Téged di\[D]csérlek,
    \[C]Bemegyek, hogy \[D]hódoljak Nek\[Em]ed.
  \endverse

  \beginverse*
    Uram, \[G]dicsér\[D]lek, d\[Am]icsérl\[Em]ek,
    Uram, \[G]dicsér\[D]lek, d\[Am]icsérl\[Em]ek!
  \endverse

  \beginchorus
    Neved áldom, \[C]szent vagy, \[D]szent vagy, Ur\[Em]am!
    A Te Neved áldom, \[C]szent vagy, \[D]szent vagy, Ur\[Em]am!
  \endchorus

\endsong

\beginsong{Csak benned}
[
  by={Géczy Katalin}
]

  \beginverse
    Csak \[Dm]Benned, csak \[C]Benned \[G] nyugszik meg \[Dm]szívem.
    Csak \[Dm]Benned, csak \[C]Benned \[G] nem \[A#]félek \[A]már.
    Csak \[F]Benned, csak \[C]Benned \[G] rejtőzöm \[Dm]szüntelen,
    Mert \[G]Jézus, Te \[C]vagy kőszik\[Dm]lám.
  \endverse

  \beginverse
    Csak ^Tőled, csak ^Tőled ^ kaphatok ^gyógyulást,
    Csak ^Tőled, csak ^Tőled ^ re^mélek ^jót.
    Csak ^Tőled, csak ^Tőled ^ nyerek fel^oldozást,
    Mert ^Jézus, Te ^vagy Megvál^tóm.
  \endverse

  \beginverse
    Csak ^Érted, csak ^Érted ^ keresztem ^felveszem,
    Csak ^Érted, csak ^Érted ^ moz^dul szí^vem.
    Csak ^Érted, csak ^Érted ^ terheim ^vállalom,
    Mert ^Nálad van ^mind, Jézu^som.
  \endverse

\endsong

\beginsong{Finn zarándokének}
[
  by={Pekka Simojoki, Koczor Tamás}
]

  \beginverse
    \[Em]Valami v\[D]éget é\[G]r, \[C]kezdődik m\[G]ás tal\[D]án,
    Látod, \[Em]megyünk az \[D/F#]élet \[G]sivatag\[Am]án,
    \[G/H]Gyere, mert \[C]lépni musz\[D]áj!
  \endverse

  \beginchorus
    Szinte \[G]hallom, \[Am] újra \[G/H]hallom, \[C] szól a \[G]régó\[D/F#]ta vándor\[Em]lók dala, \[D]
    Minden \[G]délben, \[Am] minden \[G/H]éjjel, \[C] vala\[G]hol zeng a \[D]dallam\[G]a.
  \endchorus

  \beginverse
    ^Néha söt^étség ^vár, ^arcunkba f^új a sz^él,
    De ^bátran megy^ünk, lábunk ^egyszerre l^ép.
    ^Istenünk ^eszközek^ént!
  \endverse

  \refrain

  \beginverse
    ^Sokan meg^álltak m^ár, ^elég v^olt a besz^éd,
    Egész ^életünk ^itt a ^keresztre n^éz,
    ^Értjük már ^üzenet^ét!
  \endverse

  \refrain

  \beginverse
    ^Aki sír, ^ne sírj^on, ^emeld a f^árad^tat!
    Hiszen ^minden nap ^itt van, ^velünk mar^ad,
    ^Újra ^szóljon ez a d^al!
  \endverse

  \refrain

\endsong

\beginsong{Hálóvetők}
[
  by={Marczell Márton}
]

  \beginverse\memorize
    \[D]Jöjj, jöjj hát vele\[Hm]m, nagy munka vá\[A]r!
    E\[D]vezz, én a hálót vete\[Hm]m, a jobboldalá\[A]n!
    \[G]Szólt, így szólt az Ú\[Hm]r
    -- Pedig már \[G]feladtuk, jó lesz az ú\[Hm]gy --
           %TODO
    Hogy: ``\[D/A]Menj, munkára fe\[A]l, csónakba há\[Gsus2]t!''.
  \endverse

  \beginverse
    ^Nézd, előttünk ^áll kősziklás ^út!
    ^Fáj, ha talpadba vá^g, szétmegy saru^d!
    Az ^út végén keresztfa á^ll,
    Sok ^kanyargó mérföld utá^n,
    De a ^sziklák között is talál^sz pár jó illatú\[Gsus4]t! \[G]
  \endverse

  \beginchorus
    Mert \[D]Isten nagy szobrászmes\[A]ter:
    A \[Hm]kőből is rózsát neve\[G]l,
    Mi \[D]szobrászok nem vagyu\[A]nk, csak kavicsszedő\[G]k.
    \[D]Szavában csak bízni ke\[A]ll,
    S hálónkba \[Hm]bőséges zsákmányt tere\[G]l,
    \[D]Mások közt így leszü\[A]nk hálóvet\[Gadd2]ők.
  \endchorus

  \beginverse
    ^Jöjj, csomózd meg ^jól, szakadt pár helye^n!
    ^Edzz, edzd meg karo^d, hogy biztos legye^n!
    ^Kell, hogy jobban szorít^sd,
    Hogy ^mindig javulj és javí^ts!
    Hogy ^még jobban fogjon, mint ^mikor mi akadtu\[Gsus4]nk fenn. \[G]
  \endverse

  \refrain

  \beginverse*
    /: \[D]Vesd ki most a hálót az \[A]emberek közé,
    \[Hm]Csónakunk hasítsa a \[G]tiszta tó vizét,
    \[D]Elcsitul a zápor, \[A]csendesül a szél,
    \[Hm]Jézusunk szavára a \[G]hálót vesd ki még! :/ \rep{2}
  \endverse

  \beginchorus
    Mert \[D]Isten nagy szobrászmes\[A]ter:
    A \[Hm]kőből is rózsát neve\[G]l,
    Mi \[D]szobrászok nem vagyu\[A]nk, csak kavicsszedő\[G]k.
    \[D]Szavában csak bízni ke\[A]ll,
    S hálónkba \[Hm]bőséges zsákmányt tere\[G]l,
    \[D]Mások közt így leszü\[A]nk \[G]hálóvet\[D]ők.
  \endchorus

\endsong

\beginsong{Így szólsz \\ You Say}
[
  by={Lauren Daigle, fordította: Csonka Gabriella}
]

  \ifchorded
    \beginverse*
      {\nolyrics Előjáték: \[F] \[Am] \[Dm] \[A#]}
    \endverse
  \fi

  \beginverse\memorize
    \[F]Fojtogat a gondolat, hogy \[Am]aki vagyok az kev\[Dm]és, \[A#]
    Hogy \[F]nem tudok megfelelni, \[Am]hazugság és téved\[Dm]és. \[A#]
  \endverse

  \beginverse
    ^Az lennék-e csupán, amit ^elbukom és elé^rek? ^
    ^Istenem, mondd ki vagyok, én ^csakis Benned remél^ek! ^
    % TODO: \echo{Ó-ó!} ?
  \endverse

  \beginchorus
    Te így szólsz: szeret\[F]lek, ha lehúz a ma\[Am]gány,
    Te mondod: erős \[Dm]vagy minden fáradtság u\[A#]tán!
    És mindig megtar\[F]tasz, ha olykor fela\[Am]dom,
    Ó, Istenem, tu\[Dm]dom, én hozzád tarto\[A#]zom!
    És elhi\[F]szem, \echo{Elhiszem!} ó, elhi\[Am]{sz\_{e}}m! \echo{Elhiszem!}
    Nem vagyok más Ne\[Dm]ked, \echo{Elhiszem!} mint gyerme\[A#]ked! \echo{Elhiszem!}
  \endchorus

  \beginverse
    Nem ^számít más Uram, csak hogy ^mit szól Szent Szav^ad, ^
    ^Követlek, Atyám, így ér^zem hogy mi jó s szab^ad! ^
  \endverse

  \refrain

  \beginverse
    ^Eléd tárom Istenem most ^minden bűnöm, s kincsem^et, ^
    ^Minden terhem és örömöm ^végre a tiéd leh^et. ^
  \endverse

  \refrain

  \beginverse*
    /: És elhi\[F]szem, \echo{Elhiszem!} ó, elhi\[Am]{sz\_{e}}m! \echo{Elhiszem!}
    Nem vagyok más Ne\[Dm]ked, \echo{Elhiszem!} mint gyerme\[A#]ked! \echo{Elhiszem!} :/ \rep{3}

    \[F]Elhiszem!
  \endverse

\endsong

\beginsong{Itt a szívem}
[
  by={Dobner Illés}
]

  % https://youtu.be/W2xj6v2AiIQ
  \ifchorded
    \musicnote{A közjátékok első akkordjai már a refrén utolsó szótagánál ("jön") kezdődnek.}
  \fi

  \ifchorded
    \beginverse*
      {\nolyrics Előjáték: \[E&] \[A&] \[Cm] \[A#4-A#]}
    \endverse
  \fi

  \beginverse\memorize
    \[E&] Gyenge vagyok, \[A&]ember. \[Cm] \[A#4-A#]
    \[E&] Megkavar az \[A&]élet. \[Cm] \[A#4-A#]
    \[E&] Nem tudom a \[A&]szívem \[Cm]meggyó\[A#4]gyí\[A#]tani. \[E&] \[A&] \[Cm] \[A#4-A#]
  \endverse

  \beginverse
    ^ Túl sok ami ^fájhat. ^ ^
    ^ Énnekem es ^másnak. ^ ^
    ^ Nélküled a ^lelkem ^porszem ^szél^vész^ben. ^ ^ ^
  \endverse

  \beginchorus
    Itt a \[A&]szí\[Cm]vem \[A#]kemény f\[E&]öld,
    Puszta\[A&]ság \[Cm]lett, \[A#]törd most \[E&]föl.
    Itt a \[A&]szí\[Cm]vem \[A#]száraz \[E&]kút,
    \[E&/D]Áss le \[A&]mély\[Cm]re, \[A#]törjön \[A&maj7]út
    Élő \[A&]víz\[Cm]nek, mely \[A#]Tőled jön.
  \endchorus

  \ifchorded
    \beginverse*
      {\nolyrics Közjáték: \[E&] \[A&] \[Cm] \[A#4-A#]}
    \endverse
  \fi

  \beginverse
    ^ Mindent tudok ^észből. ^ ^
    ^ Túl keveset ^szívből. ^ ^
    ^ Míg a betű ^öl, a ^Szellem ^él^ni h^ív. ^ ^ ^
  \endverse

  \beginverse*
    Refrén
  \endverse


  \ifchorded
    \beginverse*
      {\nolyrics
        Közjáték:
        \[A&] \[Cm] \[A#4-A#] \[E&-A#/D]
        % \gtab{E&}{X65(343)}, \gtab{A#/D}{XX0331}
        \[A&] \[Cm] \[A#4-A#] \[E&]
      }
    \endverse
  \fi

  \beginverse*
    /: Megtan\[A&]í\[Cm]tasz \[A#]szeretn\[E&]i, merni \[A&]él\[Cm]ni, \[A#]érez\[E&]ni. :/ \rep{3}
    \echo{Megbocsátasz, hogy én is megbocsássak.}
  \endverse

  \beginverse*
    Refrén
  \endverse

  \ifchorded
    \beginverse*
      {\nolyrics Utójáték: \[E&] \[A&] \[Cm] \[A#4-A#] \[A&]}
    \endverse
  \fi

\endsong

\beginsong{Jézus, neved oly csodálatos \\ The Sweetest Name of All}
[
  by={Tom Coomles}
]

  \beginverse\memorize
    \[C]Jézus, neved \[Dm]oly csodálatos,
    \[G]Jézus, nékem \[C]oly kívánatos,
    \[Am]Jézus, szívem \[Dm]hódoljon Neked,
    Te vagy \[G]édes mindenek fel\[C]ett!
  \endverse

  \beginverse
    ^Jézus, Sáron ^legszebb rózsája,
    ^Jézus, a földnek ^legszebb virága,
    ^Jézus, Néked ^mondok éneket,
    Te vagy ^édes mindenek fel^ett!
  \endverse

  \beginverse
    ^Jézus, eljöv^endő nagy Király,
    ^Jézus, akit ^minden szentje vár,
    ^Jézus, Tiéd ^minden tisztelet,
    Te vagy ^édes mindenek fel^ett!
  \endverse

\endsong

\beginsong{Jézus, Te vagy az Úr örökké}[]

  \beginverse
    \[Dm]Jézus, Te vagy az \[G]Úr örökké,
    \[Dm]Jézus, Te vagy a f\[G]ény!
    \[Dm]Nem bízom e vil\[G]ágban többé,
    \[Dm]Nálad \[Am]van a rem\[Dm]ény.
  \endverse

  \beginchorus
    /: \[F]Kérdésemre ig\[C]éd a válasz,
    \[A#]Hangod a vigaszta\[A]lás!
    \[Dm]Élő vizedet b\[G]őven áraszd,
    \[Dm]Szomjam \[Am]nem oltja \[Dm]más. :/ \rep{2}
  \endchorus

  \beginverse
    ^Jézus, Te vagy az ^Úr örökké!
    ^Jézus, Te vagy a f^ény!
    ^Szívem egyedül ^csak Te látod,
    ^Minde^nem a Ti^éd.
  \endverse

  \refrain

\endsong

\beginsong{Jézus, véred megtisztít}
[
  li={Kék könyv / 67.D}
]

  % https://youtu.be/9i-d8o5p3CA

  \ifchorded
    \beginverse*
      {\nolyrics Előjáték: \[C] \[Em] \[Am] \[F] \[G] \[C]}
    \endverse
  \fi

  \beginverse*
    Jézus, \[C]véred \[G]megtisz\[Am]tít, véred \[F]ad új éle\[G]tet.
    Véred \[F]tesz ma szabad\[G]dá, értem \[C]folyt e \[E\⁷]drága \[Am]vér.
    Így a \[F]lel\[Fm]kem fehér, mint a \[C]hó, \[G] mint a \[Am]hó.
    Úr \[Dm]Jézus, Te \[G]megölt \[G\⁷]Bá\[C]rány.
  \endverse

\endsong

\beginsong{Készítsük el \\ Prepare the Way}
[
  by={Don Potter}
]

  % https://youtu.be/OpwiAGq3Pk4

  \ifchorded
    \beginverse*
    %   {\nolyrics Előjáték: /: \[Hm] \[D] \[A] \[G] :/ \rep{2}}
    \endverse
  \fi

  \beginchorus
    \ifchorded
    \[Em] Készítsük el, \[C] készítsük el, \[Am] jöjjetek, készít\[Em]sük az Úr útját! \[C-D]
    \[Em] Készítsük el, \[C] készítsük el, \[Am] jöjjetek, készít\[Em]sük az Úr útját!
    \fi
    \iflyric
    /: Készítsük el, készítsük el, jöjjetek, készítsük az Úr útját! :/ \rep{2}
    \fi
  \endchorus

  \beginverse\memorize
    \[Em] Áldott a szolga, \[D]akit Mestere \[C]munkában talál, ha vissza\[Em]tér.
    \[^Em]Hamar itt a nap, és a \[D]Fiú újra jön,\[Am] lássuk meg az idő jel\[H]ét!
  \endverse

  \refrain

  \beginverse
    ^ Álljunk meg bátran a ^hit próbái közt, ^ ne sodorjon mindenféle ^szél!
    Szívembe rejtem és ^őrzöm igédet,^ úgy vágylak meglátni ^én.
  \endverse

  \refrain

  \beginverse
    ^ Kiáltó hang szól, ^ készítsetek ^ösvényt a pusztákon ^át!
    Völgy, emelkedj fel, és ^hegy, szállj alá, ^dicsőségét mindenhol ^hadd lássák.
  \endverse

  \refrain

  \ifchorded
    \beginverse*
    %   {\nolyrics Közjáték: /: \[Hm] \[G] \[D] \[A] :/ \rep{2}}
    \endverse
  \fi

  \beginverse*
    És az \[G]ég, tündök\[D]lő és \[C]fényes \[D]lesz, csak egy \[G]szempillan\[D]tás alatt \[C]történik \[D]ez.
    Eljön \[G]felhők\[D]ben és \[Am]nagy dicsőség\[D]ben fe\[Am]hér lovon \[Hm]ülve az \[C]Úris\[D]ten.
  \endverse

  \refrain

\endsong

\beginsong{Lábadnál térdelek}
[
  index={Uram, én szükségem}
  link={https://youtu.be/GpA9K9EDHGk}
]

  \transpose{2}

  \beginverse
    \[^G] \[C-G] Lábad\[G]nál \[C]térde\[G]lek, s megval\[Em]lom, \[D] itt jó Ve\[C]led.
    Ha nem len\[G]nél, a \[C]semmi \[G]vár. De szólsz U\[Em]ram: \[D] Ez jó i\[C]rány.
  \endverse

  \beginchorus
    Uram, \[G]csak rád \[C]van szük\[G]sé\[D]gem, \[Em]szünte\[C]len vár \[G]szí\[D]vem,
    Én \[G]véd\[C]elmem, m\[G]éltós\[C]ágom, U\[G]ram, én szük\[D]sé\[G]gem.
  \endchorus

  \beginverse
    ^ Kegyelmed ^győz, hol ^nő a ^bűn. Amerre ^jársz, ^ csak irga^lom.
    Krisztus^ban a ^szív sza^bad, mi bennem ^szent ^ Te vagy ma^gad.
  \endverse

  \beginverse*
    Refrén
  \endverse

  \beginverse
    \[C]Szívem \[G]dalban \[D]száll Fe\[Em]léd, a \[C]kísértésnek \[D]éjje\[C]lén,
    S mikor \[C]nincs \[G]erőm, em\[D]el az \[Em]ég, \[C]Jézus benned \[D]van re\[G]mény!
  \endverse

  \beginverse*
    Refrén \rep{2}
  \endverse

\endsong

\input{songs/menjetek-be-kapuin.tex}
\beginsong{Nagy vagy Urunk}
[
  by={Chris Tomlin / Jesse Reeves / Ed Cash | Magyar Szöveg: Marika Payne}
]

  \beginverse
    Ő \[A]tündöklő Király, \[F#m]fénye ragyog ránk,
    Hadd örvendjen a \[D2]Föld, örvendjen a Föld.
    Ő \[A]fényben lakozik, s a \[F#m]sötét eltűnik,
    Hangjára megre\[D2]meg, ha szól az megremeg.
  \endverse

  \beginchorus
    Mily’ \[A]nagy vagy Urunk, énekeljük:
    \[F#m]Nagy vagy Urunk, mind meglátjuk,
    \[Dmaj7]Nagy, mily’ n\[E]agy vagy Ur\[A]unk.
  \endchorus

  \beginverse
    ^Örökké létező, ^kezében az idő
    A kezdet és a ^vég, a kezdet és a vég.
    Az ^egyetlen Isten, ki ^Atya, Fiú, Lélek,
    A megölt Bá^rány, a Győztess Oroszlán.
  \endverse

  \refrain

  \beginverse*
    /: \[A]Méltó nagy Neved,
    \[F#m]Minden név felett,
    A \[Dmaj7]szívünk áld,
    Mily’ n\[E]agy vagy Ur\[A]unk. :/
  \endverse

  \refrain

\endsong

\beginsong{Ó, Te vagy, Uram}

  % https://youtu.be/s-vhxd1Isxk

  \ifchorded
    \beginverse*
      {\nolyrics Előjáték: \[A] \[D] \[A] \[D] \[A] \[F#m] \[C#m] \[D] \[Hm] \[D] \[E] \[E7]}
    \endverse
  \fi

  \beginchorus
    /: Ó, \[A]Te vagy U\[D]ram \[A] vágyam \[D]és örö\[A]möm, \[F#m]
    Te \[C#m]vigyázol r{\[D]\[Hm]\[D]\[E]{\_{á}}}m! \[E7] :/ \rep{2}
  \endchorus

  \beginverse\memorize
    \[A]Mikor az \[D]Úr ránkta\[A]lált,
    \[D]Úgy tűnt, mintha álmod\[A]nánk.
    A \[C]szánk megtelt neveté\[G]ssel,
    % HACK: The `\hspace{0cm}` below is used to avoid lifting up the `Hm` chord.
    És \[D]nyelvünk \[Hm]ujjon\hspace{0cm}gáss{\[G]\[E]\[E7]{\_{a}}}l!
  \endverse

  \beginverse*
    Refrén
  \endverse

  \beginverse
    ^Hatalma^sat tett vel^ünk,
    ^Szívünkből örvendezz^ünk!
    Kik ^könnyek között vet^nek,
    ^Vígadva ^kévét szedn{^^^{\_{e}}}k.
  \endverse

  \beginverse*
    Refrén \[A]
  \endverse

\endsong

\beginsong{Örvendjetek angyalok - Krisztus feltámadott}[]

  \ifchorded
    \beginverse*
      {\nolyrics Előjáték: \[D] \[A] \[D] \[A] \[Hm] \[D] \[A] \[Hm] \[A] \[F#\⁴] \[Bm]}
    \endverse
  \fi

  % \beginverse\memorize
  %   \[Dm]Örvendjetek \[F]angyalok! Jézusunk fel\[A#]támadott,
  %   \[C]És a lelkünk \[F]meglelé ösvényét a \[A#]menny felé:
  %   \[C]Alle\[A]lu\[Dm]ja, \[C]alle\[A]lu\[Dm]ja!
  % \endverse

  \beginverse\memorize
    \[Hm]Örvendjetek \[A]angya\[Hm]lok! \[D]Jézusunk fel\[A]táma\[Hm]dott,
    \[A]És a lelkünk \[D]meglelé ösvényét a \[A]menny fe\[Hm]lé: \[A]Alle\[Hm]luja!
  \endverse

  % \beginverse
  %   Dicsőítsük az Atyát, ki szent Fiát adta át:
  %   Hogy a mennynek nyerje meg a bűntépte lelkeket!
  %   Alleluja, Alleluja!
  % \endverse

  \beginverse
    ^Jó hírt hoz az ^Úr ne^künk: ^A sötétség ^már le^tűnt
    S ^nincsen rajtunk ^hatalma, ha a lelkünk ^akar^ja: ^Alle^luja!
  \endverse

  % \beginverse
  %   Hiszünk mint a gyermekek, kiknek szíve nem remeg
  %   Végtelen csodád előtt. Adj hitünknek új erőt!
  %   Alleluja, Alleluja!
  % \endverse
  %
  % \beginverse
  %   Bort és fehér kenyeret ajánlunk föl Teneked:
  %   Fogadd véle, Istenünk, Néked szentelt életünk;
  %   Alleluja, Alleluja!
  % \endverse

  \beginverse
    ^Szent vagy, szent vagy ^Jézu^sunk! ^Kősírodhoz ^elfu^tunk
    S ^angyaliddal ^Teneked zengünk boldog ^éne^ket, ^Alle^luja!
  \endverse

  % \beginverse
  %   Jézus Krisztus megjelent hófehéren idelent.
  %   Áldott Istenünk neve, ki őt nékünk küldte le!
  %   Alleluja, Alleluja!
  % \endverse
  %
  % \beginverse
  %   Jöjj szívünkbe szent kenyér, akit elménk fel nem ér:
  %   Adj szívünknek új erőt, ha lankad a vég előtt!
  %   Alleluja, Alleluja!
  % \endverse

  \beginverse
    ^Krisztus fel^táma^dott, ^kit halál el^raga^dott,
    ^Örvendezzünk, ^vígadjunk, Krisztus lett a ^víga^szunk, ^Alle^luja!
  \endverse

  \beginverse
    ^Ha ő fel ^nem tá^mad, ^nincs többé bűn^bocsá^nat,
    ^De él, ezért ^szent nevét, zengjük ő dí^csére^tét, ^Alle^luja!
  \endverse

  % \beginverse
  %   Alleluja! Alleluja! Alleluja!
  %   Örvendezzünk, vígadjunk, Krisztus lett a vígaszunk,
  %   Alleluja!
  % \endverse

  \ifchorded
    \beginverse*
      {\nolyrics Utójáték: \[Hm] \[A] \[D] \[A] \[Hm] \[D] \[A] \[Hm] \[A] \[F#\⁴] \[Bm]}
    \endverse
  \fi

\endsong

\beginsong{Röpke fohász}[]

  \beginchorus
    /: Röpke fo\[G]hászként száll a \[Em]dal égbe szá\[Hm7]rnyaló ma\[D]dár,
    Fészkét \[G]el\[D]hagyta m\[Em]ár, Hozzád ta\[C]lál. \[D] :/ \rep{2}
  \endchorus

  \beginverse
    \[G]Élsz, vagy utazol \[Hm]bárhol, de \[Em]Hozzá szól a sz\[C]ív,
    \[G]Bánat ér vagy \[Hm]öröm táncra h\[Am7]ív. \[D]
    Mint \[G]forrás tört fel a \[D]sóhaj, mint \[Em]tenger árad a \[Hm]hála.
    \[C]Hallja Ő, egy \[G]szó elég, az \[Am7]egekig felé\[D7]r!
  \endverse

  \refrain

  \beginverse
    ^Nem vagyok egyedül az ^úton, ^léptem vezeti ^Ő.
    ^Otthonom, a ^bajban segí^tőm. ^
    ^Testvérem arcán ^látom, hogy ^Jézus hű ba^rátom.
    ^Kegyelméből ^élek én, ^megnyugszom kez^én.
  \endverse

  \refrain

\endsong

\beginsong{Szentlélek}
[
  li={Kék könyv / 190. (le 4)}
]

  \beginchorus
    Szent\[D\²]lé\[D]lek, úgy \[D\²]kérünk, szállj le r\[D]ánk,
    Töltsd el a \[G]szívünk, \[Em] éle\[A]tünk,
    Hogy \[D\²]bé\[D]ke és \[D\²]áldás szálljon \[D]ránk,
    Küldd el a \[G]Lelked, úgy \[A]kérünk, Isten\[D]ünk.
  \endchorus

  \beginverse\memorize
    \[D]Kegyelmeddel \[A]táplálsz, Isten\[Hm]ünk, \[G]
    Hozzád \[Em]emeljük most mindnyájan a \[A]szívünk,
    \[D]Reménységünk \[A]Belőled fa\[Hm]kad, \[G]
    Hogyha \[Em]szívünk-lelkünk mindentől sza\[A]bad.
  \endverse

  \refrain

  \beginverse
    ^Ajándékul ^Lelked küldted e^l, ^
    Hogy ne ^önmagunknak, hanem annak ^éljünk,
    ^Aki értünk ^életét ad^ta. ^
    Mi is ^átadjuk most Néked élet^ünk.
  \endverse

  \refrain{\\Küldd el a \[G]Lelked, úgy \[A]kérünk, Isten\[D]ünk.}

\endsong

\beginsong{Szívem telve van Veled}
[
  by={Craig Musseau, Harkai Nóra},
]

  \ifchorded
    \beginverse*
      {\nolyrics Előjáték: \[E] \[G#m\⁷] \[A⁄H]}
    \endverse
  \fi

  \beginverse*
    \[E] A szívem \[G#m]telve van Ve\[C#m]led, \[C#m\⁹] drága J\[A]ézus, \[E] hű Megv\[H]áltóm.
    \[E] Hálát \[G#m]érzek mindaz\[C#m]ért, \[C#m\⁹] mit értem t\[A]ettél, \[E] drága J\[H]ézus.
    \[F#m] Hogy elhív\[C#m]tál és \[H]befogad\[C#m]tál,
    \[F#m] Nincsen \[C#m]más ottho\[H]nom.
    Karod \[E]óv és \[G#m]átöl\[A]el,
    Nálad \[E]van \[G#m]lakhely\[A]em.
    Jöjj közel \[C#m]hát és \[G#m]ölelj most \[A]át,
    Szüksé\[H]gem \[A]van {\[E]R}{\[G\shrp m C\shrp m]á}d!
    Szüksé\[H]gem van {\[E]R}{\[G\shrp m C\shrp m]á}d!
    Szüksé\[H]gem van \[E]Rád!
  \endverse

\endsong

\beginsong{Te vagy az út}
[
  by={Sillye Jenő},
  li={Dúr / 163.}
]

  % https://youtu.be/uS3Hx8wphqg

  \newchords{chorus}

  \beginverse
    \[E] Te vagy az \[H]út, az igazs\[A]ág, Te vagy az \[E]élet!
    \[E] Hozzád me\[H]gyek, áldom ne\[A]ved, általad \[E]élek.
  \endverse

  \beginchorus\memorize[chorus]
    /: \[A]Megszabadítot\[F#m]tál, \[E]kenyeret, s bort ad\[C#m]tál.
    \[E]Tested és véred t\[H]áplál, hűséged \[E]éltet. \var{1.}\[E\⁷]Alleluja! \var{2.} \transpose{1} \[H\⁷] :/ \rep{2}
  \endchorus

  \transpose{1}
  \prefersharps

  \beginverse
    ^Te vagy az ^út, az igazs^ág, Te vagy az ^élet!
    ^ Hozzád me^gyek, áldom ne^ved, általad ^élek.
  \endverse

  \beginchorus\replay[chorus]
    /: ^Megszabadítot^tál, ^kenyeret, s bort ad^tál.
    ^Tested és véred t^áplál, hűséged ^éltet. \var{1.} ^Alleluja! \var{2.} \transpose{2} ^ :/ \rep{2}
  \endchorus

  \transpose{2}

  \beginverse
    ^Te vagy az ^út, az igazs^ág, Te vagy az ^élet!
    ^ Hozzád me^gyek, áldom ne^ved, általad ^élek.
  \endverse

  \beginchorus\replay[chorus]
    /: ^Megszabadítot^tál, ^kenyeret, s bort ad^tál.
    ^Tested és véred t^áplál, hűséged ^éltet. \var{1.} ^Alleluja! \var{2.} Alleluja! :/ \rep{2}
  \endchorus

\endsong

\beginsong{Te vagy szerelme \\ You are my passion}
[
  by={Noel Richards, Tricia Richards}
]
  
  \ifchorded
    \gtab{D\²}{XX0230}
    \gtab{D/F#}{300232}
    \gtab{G\²/H}{X20033}
    \gtab{A/C#}{X42220}
  \fi

  \beginverse
    \[D]Te vagy sze\[A/C#]relme \[D/F#]életemn\[G]ek,
    \[D] Barát és \[A/C#]társ, aki s\[G\²/H]zeret. \[A/C#]
    \[D] Érinté\[A/C#]sedet \[D/F#]várja lény\[G]em,
    \[D] Teljes szí\[A/C#]vemből sze\[G\²/H]retlek!
  \endverse

  \beginchorus
    \[G\²/H] Vonj most magadhoz \[A/C#]még job\[D]ban, \[D\²]
    \[A/C#] Végy fel \[G\²/H]a ka\[A/C#]rjaidb\[D]a,
    \[G\²/H] Szíved dobbanását \[A/C#]hadd hallj\[D]am,
    \[A/C#] Ó, \[G\²/H] \[A/C#] Jéz\[D]us,
    \[A/C#] Ó, \[G\²/H] \[A]Jéz\[D]us!
  \endchorus

\endsong

\beginsong{Több erőt, több szeretetet \\ More Love, More Power}
[
  by={Jude del Hierro, Simonfalvi Zsolt}
]

  % Contributor: Korcsma Eszter

  % TODO: This song has about 4-5 different versions in terms of variations of
  % chords, ask Márton Marczell which should be the canonical Hungarian one.

  \beginverse*
    /: \[Em] Több erőt, \[C] több szeretetet,
    \[D] Többet \[Hm]Tőled minden \var{1.} \[Em]nap! \[Em] \[D] \var{2.} \[Em]nap! :/ \rep{2}
  \endverse

  \beginchorus
    És teljes szívem\[Am]ből i\[D]mádom Nev\[Em]ed,
    Az elmém meghaj\[Am]ol és \[D]hódol Nek\[Em]ed,
    Teljes lényem Ti\[Am]éd, ren\[D]delkezz vel\[Em]em,
    Mert Te vagy az \[C]Úr! \[D] Te vagy az \[Em]Úr.
  \endchorus

\endsong

\beginsong{Utazós dal}
[
  by={Szabó András}
]

  % https://youtu.be/eAXXsMdLEdA
  % https://youtu.be/E6oXu26MOAE

  \transpose{2}

  \beginverse
    Ha a \[C]Lelked viszi \[F]léptem rögös \[Am]ösvényei\[G]den,
    Jöhet \[C]tűz, ár, jöhet \[F]orkán, az u\[Am]tam nem \[G]vész \[C]el.
  \endverse

  \beginverse
    Magas ^erdő, színes ^égbolt hajol ^ösvényem^re,
    Ez a ^szépség új ^háladalt ^ír szí^vem^be.
  \endverse

  \beginchorus
    Ez az \[C⁄E]út, mit nekem \[F]adtál gyönyö\[Am]rű s néha \[G]fáj,
    De e\[C]lőttem a ke\[F]reszted és a \[Am]csilla\[G]god \[C]már.
  \endchorus

  \beginverse
    Ha az ^éjjel hideg ^árnya lepi ^ösvényem ^el
    Csak a ^Hajnalcsil^lagfény ragyog ^és ve^zér^el.
  \endverse

  \beginverse
    Szaka^dékból helyes ^útra aki ^elvezet^tél,
    A ve^zérem, úti^társam, úti^célom ^let^tél.
  \endverse

  \refrain

  \beginverse*
    /: Veled \[Am]lendüljön a \[F]léptünk, Neked \[C]zendüljön a \[G]szívünk,
    Veled \[Am]lendüljön a \[F]léptünk, ó \[C]hű Vezé\[G]rünk! :/ \rep{2}
  \endverse

  \refrain

  \beginchorus
    Ez az \[C⁄E]út, mit nekem \[F]adtál gyönyö\[Am]rű s néha \[G]fáj,
    De e\[C]lőttem új \[F]égbolt és \[Am]új föld \[G]mi \[C]vár.
  \endchorus

\endsong

\beginsong{Vedd a szívem \\ I surrender}

  \versesep=12pt plus 3pt minus 12pt

  \ifchorded
    \gtab{Dm\⁷}{XX0211}
    \gtab{F\ᵐᵃʲ\⁷}{102210}
  \fi

  \ifchorded
    \beginverse*
       {\nolyrics Előjáték: /: \[Am] \[C] \[G] \[F\ᵐᵃʲ\⁷] :/ \rep{2}}
    \endverse
  \fi

  \beginverse
    \[Am]Eljöttem, hogy meghajtsam \[C]térdemet. Tiéd va\[G]gyok! Tiéd va\[F\ᵐᵃʲ\⁷]gyok!
    \[Am]Várj most rám, ölelj át, \[C]vágyom Rád! Éhezem \[G]Rád! Éhezem \[F\ᵐᵃʲ\⁷]Rád!
    Vedd a \[Am]szívem! \[C] \[G] \[F\ᵐᵃʲ\⁷]
  \endverse
  \beginverse
    \[Am]Irgalmad, kegyelmed \[C]járjon át! Szomjazom R\[G]ád! Szomjazom \[F\ᵐᵃʲ\⁷]Rád!
    Ki\[Am]tárt karral, könnyek közt \[C]áldalak! Szólj most hozz\[G]ám! Szólj most ho\[F\ᵐᵃʲ\⁷]zzám!
  \endverse

  \beginchorus
    /: Vedd a \[Am]szívem! Vedd a \[C]szívem!
    Vágyom Rád \[Dm\⁷]még jobban! Vágyom Rád \[F]még jobban! :/ \rep{2}
  \endchorus

  \ifchorded
    \beginverse*
       {\nolyrics \[F] \[C] \[G] \[Dm\⁷] \[Am] \[G]}
    \endverse
  \fi

  \def \shrink {\ifchorded\hspace{-0.25cm}\fi}

  \beginverse*
    /: Mint egy \[F]zúgó \[C]szél, Jézus \[G]bennem él!\
    Uram Ti\[Dm\⁷]éd, \shrink Uram, Ti\[Am]éd \shrink min\[G]den!
    Mint egy \[F]nagy vi\[C]har, Jézus \[G]felkavar!\
    Uram Ti\[Dm\⁷]éd, \shrink Uram, Ti\[Am]éd \shrink min\[G]den! :/ \rep{2}
    Uram Ti\[Dm\⁷]éd, Uram, Ti\[Am]éd mind\[G]en!
  \endverse

  \refrain

  \beginverse*
    Vedd a \[Am]szívem!
  \endverse

\endsong

\beginsong{Zengd velünk}
[
  by={Gocam ??? Varga Attila},
  li={Kék könyv / 230. [-2]}
]

  \transpose{-2}

  \beginchorus
    /: \[Hm]Alleluja, \[A]alleluja, \[G]allelu\[F#]ja! :/ \rep{2}
  \endchorus

  \beginverse
    \[D]Zengd velünk: örökké \[A]jó az Úr, \[Hm]énekeld szerete\[F#]tét!
    \[D]Izrael háza is \[A]mondja hát, \[Hm]zengje \[G]hát nagy örö\[F#]mét!
  \endverse

  \refrain

  \beginverse
    Az ^Úr velünk, ki lehet ^ellenünk? ^Rettegést nem isme^rünk.
    ^Jobb az Úr ereje, ^éneke, ^jobb is, ^mint az embe^ré!
  \endverse

  \refrain

  \beginverse
    ^Ő az én teremtő ^Istenem, ^Istenem, ki nem hagy ^el.
    ^Benne én ujjongva ^járhatok; ^élhe^tek a tenye^rén!
  \endverse

  \refrain

\endsong



    % \setcounter{songnum}{999}
    % \beginsong{Alleluja}[]

    %   \beginchorus
    %     Alleluja, alleluja…
    %   \endchorus

    %   \beginverse*
    %     Készítsétek elő az Úr útját!
    %     Egyengessétek ösvényét,
    %     És minden ember meglátja az Üdvözítőt, akit elküld az Isten.
    %   \endverse

    %   \beginchorus
    %     Alleluja, alleluja…
    %   \endchorus

    % \endsong

    % \setcounter{songnum}{999}
    % \beginsong{Alleluja}
    % [
    %   by={Gocam ??? Varga Attila},
    %   li={Kék könyv / 230. [-2]}
    % ]

    %   \beginchorus
    %     /: \[Hm]Alleluja, \[A]alleluja, \[G]allelu\[F#]ja! :/ \rep{2}
    %   \endchorus

    %   \beginverse
    %     \[D]Ó, Urunk, mutassad \[A]meg nekünk, \[Hm]irgalmas szíve\[F#]det,
    %     \[D]És az üdvösséget \[A]add nekünk, \[Hm]add meg \[G]nékünk, Iste\[F#]nünk!
    %   \endverse

    %   \beginchorus
    %     /: \[Hm]Alleluja, \[A]alleluja, \[G]allelu\[F#]ja! :/ \rep{2}
    %   \endchorus

    % \endsong

    % \renewcommand{\thesongnum}{T\arabic{songnum}}
    % \setlength{\songnumwidth}{1.1cm}
    \renewcommand{\thesongnum}{T\arabic{songnum}}
\setlength{\songnumwidth}{1.15cm}
\setcounter{songnum}{1}

\input{songs-taize/ahol-szeretet.tex}
\input{songs-taize/aldott-legy-uram.tex}
\input{songs-taize/bizakodjatok-jo-az-ur.tex}
\beginsong{Csak vándorolunk}
[
  by={Jacques Berthier}
]

  \beginverse*
    Csak \[Dm]vándorolunk az \[A#]éjben, mert \[C\⁶]forrás vi\[Gm/A#]zére \[A\⁴]vá\[A]gyunk.
    \[Dm]Szomjunk a \[C]fény a sö\[F]{t\_{é}}t\[A]ben, szomjunk a \[A#]fény a sö\[A]tétben.
  \endverse

\endsong

\beginsong{Gyújts éjszakánkba fényt}
[
  by={Jacques Berthier}
]

  \beginverse*
    \[H]Gyújts éjszakánkba \[Em]fényt, hadd égjen a
    Soha ki nem al\[D]vó \[G]tűz, a ki nem \[C]al\[G]vó \[D]tűz!
    Gyújts \[G]éjszakánkba \[C]fényt, hadd égjen a
    \[H]Soha ki nem \[Em]al\[Am\⁶]vó \[H]tűz, a ki nem \[Em]al\[Am\⁶]vó \[H]tűz!
  \endverse

\endsong

\input{songs-taize/irgalmas-istenunk.tex}
\input{songs-taize/jezus-eletem.tex}
\beginsong{Jézus, majd gondolj rám}
[
  by={Jacques Berthier}
]

  \beginverse*
    \[D]Jézus, majd \[Em\⁷]gondolj rám, \[A]ha a Te országod \[D]{\_{e}}ljön.
    \[Hm]Jézus, majd \[Em]gondolj rám, \[A]ha a Te országod \[D]eljön.
  \endverse

\endsong

\input{songs-taize/jo-az-urban-bizakodni.tex}
\beginsong{Ne félj, ne aggódj}
[
  by={Jacques Berthier}
]

  \beginverse*
    \[Am]Ne félj, ne \[Dm\⁷]aggódj, \[G]ne sírj, ne \[Cmaj\⁷]bánkódj:
    \[F]Ha tiéd \[Dm\⁶]Isten, \[E]tiéd már \[Am]minden.
    \[Am]Ne félj, ne \[Dm\⁷]aggódj, \[G]ne sírj, ne \[Cmaj\⁷]bánkódj:
    \[F]Elég \[Dm\⁶]Ő \[E]né\[Am]ked.
  \endverse

\endsong

\input{songs-taize/urra-var-a-szivunk.tex}
\beginsong{Várj és ne félj}
[
  by={Jacques Berthier}
]

  \beginverse*
    \[Em]Várj és ne \[C]félj, az \[Am\⁶]Úr jön \[H\⁷]már!
    \[Em]Várj \[D]és ne \[G]félj, hű \[Am\⁷]szív\[H]vel \[Em]várj!
  \endverse

\endsong


\setlength{\songnumwidth}{1.45cm}
\setcounter{songnum}{101}

\beginsong{Alleluja}
[
  li={Kék könyv / 243. [-3]} % TODO: Check -3?
]

  \beginverse*
    Alle\[Hm]luja, alle\[D]luja, alle\[G]l\[A]{\_{u}}\[D]ja.
    Alle\[Hm]luja, alle\[D]luja, alle\[G]l\[Em F#]{\_{u}}\[Hm]ja!
  \endverse

\endsong



    % \renewcommand{\thesongnum}{K\arabic{songnum}}
    \renewcommand{\thesongnum}{K\arabic{songnum}}
\setlength{\songnumwidth}{1cm}
\setcounter{songnum}{1}

\input{songs-extra/hala-neked-istenunk.tex}
\input{songs-extra/koszonjuk-neked-urunk.tex}
\beginsong{Jó Atyánk, köszönjük Néked}[]

  \ifchorded
    \beginverse*
      {\nolyrics Előjáték: /: \[D] \[D\⁴] \[D] \[D\²] :/ \rep{2}}
    \endverse
  \fi

  \beginverse*\memorize
    \[D]Jó Atyánk, köszönjük Néked \[G]_____-t,
    Szívünkbe zártuk, \[D]hála \[D\⁴]érte! \[D] \[D\²]
  \endverse

  \beginverse*
    ^Áldj meg minket, Krisztusunk,
    A ^Lelked legyen közöttünk míg ^együtt ^vagyunk! ^ ^
  \endverse

  \ifchorded
    \beginverse*
      {\nolyrics \[H\⁷]}
    \endverse
  \fi

  \beginverse*\memorize
    \[E]Ó, Urunk, mi \[A]egyet aka\[E]runk,
    Kik ebben a dalban \[H\⁷]együtt vagyunk:
  \endverse

  \beginverse*
    ^Szereteted fénye, á^ldása érje ^_____ -t!
    Legyen a ^Lelked véle! \[E]Alle\[A]lu\[E]ja!
  \endverse

\endsong



    % \renewcommand{\thesongnum}{M\arabic{songnum}}
    % \setlength{\songnumwidth}{1.25cm}
    \renewcommand{\thesongnum}{M\arabic{songnum}}
\setlength{\songnumwidth}{1.3cm}

\setcounter{songnum}{10}
\beginsong{Uram, irgalmazz I.}
[
  by={Sillye Jenő},
  li={Kék könyv / 159.}
]

  \ifchorded
    \beginverse*
      {\nolyrics
        Előjáték:
        /: \[Dm] \[Gm] \[C\⁷] \[F] :/ \rep{2}
        \[Dm] \[Gm] \[A\⁷] \[Dm]
      }
    \endverse
  \fi

  \beginverse*
    \[Dm] Uram, irgal\[Gm]mazz! \[C\⁷] Uram, irgal\[F]mazz!
    \[Dm] Krisztus, kegyel\[Gm]mezz! \[C\⁷] Krisztus, kegyel\[F]mezz!
    \[Dm] Uram, irgal\[Gm]mazz! \[A\⁷] Uram, irgal\[Dm]mazz!
  \endverse

  \ifchorded
    \beginverse*
      {\nolyrics Utójáték:
        \[Dm] \[Gm] \[C\⁷] \[F]
        \[Dm] \[Gm] \[A\⁷] \[Dm]
      }
    \endverse
  \fi

\endsong

\beginsong{Uram, irgalmazz II.}
[
  index={Jézus Krisztus, Téged elküldött az Atya},
  li={Kék könyv / 129.B}
]

  \ifchorded
    \beginverse*
      {\nolyrics Előjáték: \[A\⁶] \[G\⁶] \[A] \[G-D]}
    \endverse
  \fi

  \beginverse\memorize
    \[A\⁶]Jézus Krisztus, Téged \[G\⁶]elküldött az Atya,
    Hogy \[A]gyógyítsd a megtört \[D]szívűeket. \[D\⁷]
    /: Uram, \[G]irgalmazz nekünk!
    Uram, \[D]irgalmazz ne\[Hm]künk!
    Uram, \[Em]irgal\[A]mazz nek\[G]ünk! \[D] :/ \rep{2}
  \endverse

  \beginverse
    ^Jézus Krisztus, Te ^eljöttél hozzánk,
    Hogy ^hívjad a bűnös ^embereket. ^
    /: Krisztus, ^kegyelmezz nekünk!
    Krisztus, ^kegyelmezz ne^künk!
    Krisztus, ^kegyel^mezz nek^ünk! ^ :/ \rep{2}
  \endverse

  \beginverse
    ^Jézus Krisztus, Te az ^Atya jobbján ülsz,
    Hogy ^értünk örökre ^közbenjárj. ^
    /: Uram, ^irgalmazz nekünk!
    Uram, ^irgalmazz ne^künk!
    Uram, ^irgal^mazz nek^ünk! ^ :/ \rep{2}
  \endverse

\endsong

\beginsong{Uram, irgalmazz III.}
[
  li={Dúr / 43.}
]

  \beginverse*
    \[A] Uram, \[D]irgal\[E]mazz! \[A] Uram, \[D]irgal\[E]mazz!
    \[G] Krisztus, \[D]kegyel\[E]mezz! \[G] Krisztus, \[D]kegyel\[E]mezz!
    \[A] Uram, \[D]irgal\[E]mazz! \[A] Uram, \[D]irgal\[E]mazz!
  \endverse

\endsong


\setcounter{songnum}{20}
\beginsong{Dicsőség I.}
[
  by={Sillye Jenő},
  li={Kék könyv / 159.}
]

  \beginverse*
    \[F] \[F]Dicsőség a \[Gm]magasságban \[C7] Isten\[F]nek,
    \[F]És a Földön \[Gm]békesség a \[C7]jószándékú \[Dm]embernek! \[Dm]
    \vspace{0.2cm}
    \[A#]Dicsőítünk \[C7]Téged, \[Gm] áldunk \[Dm]Téged,
    \[A#] Imádunk \[C7]Téged, \[Gm]magasztalunk \[Dm]Téged,
    \vspace{0.2cm}
    \[C7]Hálát adunk \[F]Neked \[Gm] nagy dicső\[F]ségedért,
    \[F] Urunk és \[Gm]Istenünk, \[C7] mennyei Ki\[F]rály,
    \[Gm]Mindenható \[C7]Atyais\[F]ten! \[Gm] \[C7] \[F]
    \vspace{0.2cm}
    \[A#]Urunk, Jézus \[C7]Krisztus, \[Gm]egyszülött Fi\[Dm]ú,
    \[A#] Urunk és \[C7]Istenünk, \[Gm] Isten \[Dm]Báránya,
    \vspace{0.2cm}
    Az \[C7]Atyának Fi\[F]a, Te elveszed a \[E]világ bűne\[A]it, \[A] \[Dm] irgalmazz \[Gm]nékünk; \[Gm]
    Te \[F]elveszed a \[E]világ bűne\[A]it, \[A] \[Dm] hallgasd meg könyörgésün\[Gm]ket.
    Te az \[F]Atya jobbján \[E]ülsz, \[E] irgalmazz \[A]nékünk!
    Mert \[D]egyedül Te vagy a Szent, Te vagy az \[A]Úr,
    Te vagy az \[F#m]egyetlen Föl\[E]ség, Jézus \[A]Krisztus,
    A \[F#]Szentlélekkel \[Hm]együtt, az \[E]Atyaisten dicsőségé\[A]ben. \[G]
    \[A]{\_{A}}\[D]men.
  \endverse

\endsong

\beginsong{Dicsőség II.}
[
  by={M. Wittal},
  li={Dúr / 36.}
]

  \versesep=12pt plus 3pt minus 10pt

  \ifchorded
    \beginverse*
      {\nolyrics Előjáték: \[D] \[A] \[D]}
    \endverse
  \fi

  \beginchorus
    \ifchorded
      /: \[D]Gloria in ex\[G]celsis \[A]De\[D]o!\
      \[C]Gloria Deo \[G]Domi\var{1.}-\[D]no! \[Am]\var{2.}-\[D]no! :/ \rep{2}
    \fi
    \iflyric
      /: Gloria in excelsis Deo! Gloria Deo Domino! :/ \rep{2}
    \fi
  \endchorus

  \beginverse\memorize
    \[D]És a \[A]földön \[F#m]békes\[Hm]ség a \[G]jószándékú \[A]embereknek!
    \[D]Dicsőítünk \[A]Téged, \[Hm]áldunk Téged, \[^F#m]imádunk \[^G]Téged, \[^D]magasztalunk Téged,
    \[C]Hálát adunk \[A]Néked \[C]nagy dicsősége\[A]dért!
  \endverse

  \refrain

  \beginverse
    ^Urunk és ^Istenünk, ^mennyei ^Király, ^mindenható Atya^isten.
    ^Urunk, ^Jézus ^Krisztus, \[Em]egyszülött Fiú, \[C]Urunk és Iste\[A]nünk,
    ^ Isten Bárá^nya, az ^Atyának Fi^a.
  \endverse

  \refrain

  \beginverse
    Te ^elveszed a ^világ ^bűne^it, ^ irgalmazz ^nékünk!
    Te \[F#m]elveszed a \[Hm]világ \[G]bűne\[D]it, \[G]hallgasd meg könyörgé\[A]sünket!
    ^Te, aki az ^Atyának ^jobbján \[G]ülsz, \[C] irgalmazz \[A]nékünk!
  \endverse

  \refrain

  \beginverse
    ^Mert egye^dül ^Te vagy a ^Szent, \[F#m]Te vagy az \[Hm]Úr,
    ^Te vagy az egyetlen ^Fölség, \[G]Jézus Krisz\[A]tus,
    A ^Szentlé^lekkel ^együtt, az \[F#m]Atyaisten \[G]dicsősé\[D]gében. ^A^men.
  \endverse

\endsong


\setcounter{songnum}{30}
\beginsong{Szent vagy I.}
[
  by={Illésy István},
  li={Kék könyv / 192.}
]

  \ifchorded
    \beginverse*
      {\nolyrics Előjáték: \[F] \[Em] \[Am] \[Dm] \[G] \[G\⁷] \[C]}
    \endverse
  \fi

  \beginverse\memorize
    Szent vagy, \[C]szent vagy, szent vagy, \[G\⁷]sz{\_{e}}nt vagy,
    Minden\[F]ségnek Ura, \[G]Istene, \[C]
    \[C\⁷] Dicső\[F]séged betölti a menny\[Em]et és a Föld\[Am]et,
    Hozsan\[Dm]na \[G] a magas\[G\⁷]ság\[C]ban!
  \endverse

  \beginverse
    Szent vagy, ^szent vagy, szent vagy, ^szent vagy,
    Minden^ségnek Ura, ^Istene, ^
    ^ \_{Ó} ^áldott, ki az Úr nevében ^eljön mihoz^zánk,
    Hozsan^na ^ a magas^ság^ban!
  \endverse

  \ifchorded
    \beginverse*
      {\nolyrics Utójáték: \[F] \[Em] \[Am] \[Dm] \[G] \[G\⁷] \[C]}
    \endverse
  \fi

\endsong

\beginsong{Szent vagy II.}
[
  by={Sillye Jenő},
  li={Kék könyv / 159. | Dúr / 93.}
]

  \beginverse*
    \[C7]
    Szent \[F]vagy, \[C7] szent \[F]vagy, \[A#]szent vagy, \[C7]mindenség Ura, \[F]Istene! \[C7]
    Dicső\[F]séged \[C7] betöl\[F]ti a \[A#]mennyet \[C7]és a föl\[F]det.
  \endverse

  \beginchorus
    /: \[C7]Hozsanna a magasság\[F]ban! \[Gm]Áldott, aki \[C7]jön \[F]Úr nevé\[Dm]ben. :/ \rep{2}
  \endchorus

  \beginverse*
    \[C7]Hozsanna a magasság\[F]ban!
  \endverse

\endsong


\setcounter{songnum}{40}
\beginsong{Isten báránya I.}
[
  by={Ferenczy Rudolf (Dax)},
  li={Kék könyv / 299.}
]

  \ifchorded
    \beginverse*
      {\nolyrics Előjáték: \[Em] \[Am] \[E&\ᵈⁱᵐ] \[Em]}
    \endverse
  \fi

  \beginverse
    \[Em]Bűneinket, égi Bárány, \[Am]szent vérednek \[D\⁷]drága árán
    \[G]Mind elveszed, \[Am]irgal\[H\⁷]mazz ne\[Em]kem!
  \endverse

  \beginverse
    \[Em]Bűneinket, égi Bárány, \[Am]szent vérednek \[D\⁷]drága árán
    \[G]Mind elveszed, \[Am]irgal\[H\⁷]mazz ne\[Em]künk!
  \endverse

  \beginverse
    \[Em]Bűneinket, égi Bárány, \[Am]szent vérednek \[D\⁷]drága árán
    \[G]Mind elveszed, \[Am]adj bé\[H\⁷]két ne\[Em]künk!
  \endverse

  \ifchorded
    \beginverse*
      {\nolyrics Utójáték: \[Em] \[Am] \[E&\ᵈⁱᵐ] \[Em]}
    \endverse
  \fi

\endsong

\beginsong{Isten báránya II.}
[
  li={Kék könyv / 302.B}
]

  \beginverse\memorize
    \[Em]Isten báránya, Te \[D]elveszed a \[C]világ bűne\[Em]it,
    \[A]Irgalmazz, \[C]irgalmazz ne\[Em]künk! \[Em\⁷] \[A] \[Am]
  \endverse

  \beginverse
    ^Isten báránya, Te ^elveszed a ^világ bűne^it,
    ^Irgalmazz, ^irgalmazz ne^künk! ^ ^ ^
  \endverse

  \beginverse
    ^Isten báránya, Te ^elveszed a ^világ bűne^it,
    ^Adj nekünk, ^adj nekünk bé^két! ^ ^ ^ \[Em]
  \endverse

\endsong

\beginsong{Isten báránya III.}
[
  by={Sillye Jenő},
  li={Kék könyv / 159. | Dúr / 94.}
]

  \beginverse
    \[Dm]Isten bárá\[Gm]nya, Te \[A#]elveszed a világ bűne\[F]it,
    \[Gm]Irgalmazz nekü\[C7]nk, \[F]irgalmazz nekü\[Dm]nk, \[Gm]irgalmazz nekü\[A]nk! \[Dm]
  \endverse

  \beginverse
    ^Isten bárá^nya, Te ^elveszed a világ bűne^it,
    ^Irgalmazz nekü^nk, ^irgalmazz nekü^nk, ^irgalmazz nekü^nk! ^
  \endverse

  \beginverse
    ^Isten bárá^nya, Te ^elveszed a világ bűne^it,
    ^Adj nekünk bék^ét, ^adj nekünk bék^ét, ^adj nekünk bék^ét! ^
  \endverse

\endsong


  \end{songs}
  % \newpage
  % \thispagestyle{empty}
  % % To be continued...
  % \vspace*{\fill}
  % % \vfill
  % Budaörs, 2020. január 02.
\end{document}
