\beginsong{Itt a szívem}
[
  by={Dobner Illés}
]

  % https://youtu.be/W2xj6v2AiIQ

  % \gtab{E&}{X65(343)}, \gtab{A#/D}{XX0331}

  \ifchorded
    \musicnote{A közjátékok első akkordjai már a refrén utolsó szótagánál ("jön") kezdődnek.}
  \fi

  \ifchorded
    \beginverse*
      {\nolyrics Előjáték: \[E&] \[A&] \[Cm] \[A#4-A#]}
    \endverse
  \fi

  \beginverse\memorize
    \[E&] Gyenge vagyok, \[A&]ember. \[Cm] \[A#4-A#]
    \[E&] Megkavar az \[A&]élet. \[Cm] \[A#4-A#]
    \[E&] Nem tudom a \[A&]szívem \[Cm]meggyó\[A#4]gyí\[A#]tani. \[E&] \[A&] \[Cm] \[A#4-A#]
  \endverse

  \beginverse
    ^ Túl sok ami ^fájhat. ^ ^
    ^ Énnekem es ^másnak. ^ ^
    ^ Nélküled a ^lelkem ^porszem ^szél^vész^ben. ^ ^ ^
  \endverse

  \beginchorus
    Itt a \[A&]szí\[Cm]vem \[A#]kemény f\[E&]öld,
    Puszta\[A&]ság \[Cm]lett, \[A#]törd most \[E&]föl.
    Itt a \[A&]szí\[Cm]vem \[A#]száraz \[E&]kút,
    \[E&/D]Áss le \[A&]mély\[Cm]re, \[A#]törjön \[A&maj7]út
    Élő \[A&]víz\[Cm]nek, mely \[A#]Tőled jön.
  \endchorus

  \ifchorded
    \beginverse*
      {\nolyrics Közjáték: \[E&] \[A&] \[Cm] \[A#4-A#]}
    \endverse
  \fi

  \beginverse
    ^ Mindent tudok ^észből. ^ ^
    ^ Túl keveset ^szívből. ^ ^
    ^ Míg a betű ^öl, a ^Szellem ^él^ni h^ív. ^ ^ ^
  \endverse

  \refrain

  \ifchorded
    \beginverse*
      {\nolyrics
        Közjáték:
        \[A&] \[Cm] \[A#4-A#] \[E&-A#/D]
        \[A&] \[Cm] \[A#4-A#] \[E&]
      }
    \endverse
  \fi

  \beginverse*
    /: Megtan\[A&]í\[Cm]tasz \[A#]szeretn\[E&]i, merni \[A&]él\[Cm]ni, \[A#]érez\[E&]ni. :/ \rep{3}
    \echo{Megbocsátasz, hogy én is megbocsássak.}
  \endverse

  \refrain

  \ifchorded
    \beginverse*
      {\nolyrics Utójáték: \[E&] \[A&] \[Cm] \[A#4-A#] \[A&]}
    \endverse
  \fi

\endsong
