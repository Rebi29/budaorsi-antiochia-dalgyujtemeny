\beginsong{Hálóvetők}
[
  by={Marczell Márton}
]

  \beginverse\memorize
    \[D]Jöjj, jöjj hát vele\[Hm]m, nagy munka vá\[A]r!
    E\[D]vezz, én a hálót vete\[Hm]m, a jobboldalá\[A]n!
    \[G]Szólt, így szólt az Ú\[Hm]r
    -- Pedig már \[G]feladtuk, jó lesz az ú\[Hm]gy --
           % TODO:
    Hogy: ``\[D/A]Menj, munkára fe\[A]l, csónakba há\[G\ˢᵘˢ\²]t!''.
  \endverse

  \beginverse
    ^Nézd, előttünk ^áll kősziklás ^út!
    ^Fáj, ha talpadba vá^g, szétmegy saru^d!
    Az ^út végén keresztfa á^ll,
    Sok ^kanyargó mérföld utá^n,
    De a ^sziklák között is talál^sz pár jó illatú\[G\ˢᵘˢ\⁴]t! \[G]
  \endverse

  \beginchorus
    Mert \[D]Isten nagy szobrászmes\[A]ter:
    A \[Hm]kőből is rózsát neve\[G]l,
    Mi \[D]szobrászok nem vagyu\[A]nk, csak kavicsszedő\[G]k.
    \[D]Szavában csak bízni ke\[A]ll,
    S hálónkba \[Hm]bőséges zsákmányt tere\[G]l,
    \[D]Mások közt így leszü\[A]nk hálóvet\[G\ᵃᵈᵈ\²]ők.
  \endchorus

  \beginverse
    ^Jöjj, csomózd meg ^jól, szakadt pár helye^n!
    ^Edzz, edzd meg karo^d, hogy biztos legye^n!
    ^Kell, hogy jobban szorít^sd,
    Hogy ^mindig javulj és javí^ts!
    Hogy ^még jobban fogjon, mint ^mikor mi akadtu\[G\ˢᵘˢ\⁴]nk fenn. \[G]
  \endverse

  \refrain

  \beginverse*
    /: \[D]Vesd ki most a hálót az \[A]emberek közé,
    \[Hm]Csónakunk hasítsa a \[G]tiszta tó vizét,
    \[D]Elcsitul a zápor, \[A]csendesül a szél,
    \[Hm]Jézusunk szavára a \[G]hálót vesd ki még! :/ \rep{2}
  \endverse

  \beginchorus
    Mert \[D]Isten nagy szobrászmes\[A]ter:
    A \[Hm]kőből is rózsát neve\[G]l,
    Mi \[D]szobrászok nem vagyu\[A]nk, csak kavicsszedő\[G]k.
    \[D]Szavában csak bízni ke\[A]ll,
    S hálónkba \[Hm]bőséges zsákmányt tere\[G]l,
    \[D]Mások közt így leszü\[A]nk \[G]hálóvet\[D]ők.
  \endchorus

\endsong
